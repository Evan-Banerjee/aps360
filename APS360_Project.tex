\documentclass{article} % For LaTeX2e
\usepackage{iclr2022_conference,times}
% Optional math commands from https://github.com/goodfeli/dlbook_notation.
\input{math_commands.tex}

%######## APS360: Uncomment your submission name
\newcommand{\apsname}{Project Proposal}
%\newcommand{\apsname}{Progress Report}
%\newcommand{\apsname}{Final Report}

%######## APS360: Put your Group Number here
\newcommand{\gpnumber}{20}

\usepackage{hyperref}
\usepackage{url}
\usepackage{graphicx}

%######## APS360: Put your project Title here
\title{Formatting Instructions for APS360 Project  \\ 
based on ICLR Conference Format}


%######## APS360: Put your names, student IDs and Emails here
\author{Evan Banerjee  \\
Student\# 1009682309\\
\texttt{evan.banerjee@mail.utoronto.ca} \\
\And
Diego Ciudad Real Escalante  \\
Student\# 1009345308 \\
\texttt{diego.ciudadrealescalante@mail.utoronto.ca} \\
\AND
Noah Monti  \\
Student\# 1009452398 \\
\texttt{noah.monti@mail.utoronto.ca} \\
\And
Ji Hong Sayo \\
Student\# 1007314728 \\
\texttt{ji.sayo@mail.utoronto.ca} \\
\AND
}

% The \author macro works with any number of authors. There are two commands
% used to separate the names and addresses of multiple authors: \And and \AND.
%
% Using \And between authors leaves it to \LaTeX{} to determine where to break
% the lines. Using \AND forces a linebreak at that point. So, if \LaTeX{}
% puts 3 of 4 authors names on the first line, and the last on the second
% line, try using \AND instead of \And before the third author name.

\newcommand{\fix}{\marginpar{FIX}}
\newcommand{\new}{\marginpar{NEW}}

\iclrfinalcopy 
%######## APS360: Document starts here
\begin{document}


\maketitle

\begin{abstract}
This template should be used for all your project related reports in APS360 course. -- Write an abstract for your project here. Please review the \textbf{ First Course Tutorial} for a quick start
%######## APS360: Do not change the next line. This shows your Main body page count.
----Total Pages: \pageref{last_page}
\end{abstract}



\section{Project Document Submission for APS360 Course}


The format for the submissions is a variant of the ICLR 2022 format.
Please read carefully the instructions below, and follow them
faithfully. There is a \textbf{9 page} limit for the main text. References do not have any limitation. This is also ICLR's standard length for a paper submission. 
If your main text goes to page 10, a $-20\%$ penalty would be applied. If your main text goes to page 11, you will not receive any grade for your submission. 

\subsection{Style}

Papers to be submitted to APS360 must be prepared according to the
instructions presented here.

Authors are required to use the APS360 \LaTeX{} style files obtainable at the
APS360 website on Quercus. Tweaking the style is not permitted.

\subsection{Retrieval of style files}

The file \verb+APS360_Project.pdf+ contains these
instructions and illustrates the various formatting requirements your APS360 paper must satisfy.
Submissions must be made using \LaTeX{} and the style files
\verb+iclr2022_conference.sty+ and \verb+iclr2022_conference.bst+ (to be used with \LaTeX{}2e). The file
\verb+APS360_Project.tex+ may be used as a ``shell'' for writing your paper. All you have to do is replace the author, title, abstract, and text of the paper with
your own.

The formatting instructions contained in these style files are summarized in
sections \ref{gen_inst}, \ref{headings}, and \ref{others} below.

\section{General formatting instructions}
\label{gen_inst}

The text must be confined within a rectangle 5.5~inches (33~picas) wide and
9~inches (54~picas) long. The left margin is 1.5~inch (9~picas).
Use 10~point type with a vertical spacing of 11~points. Times New Roman is the
preferred typeface throughout. Paragraphs are separated by 1/2~line space,
with no indentation.

Paper title is 17~point, in small caps and left-aligned.
All pages should start at 1~inch (6~picas) from the top of the page.

Authors' names are
set in boldface, and each name is placed above its corresponding
address. The lead author's name is to be listed first, and
the co-authors' names are set to follow. Authors sharing the
same address can be on the same line.

Please pay special attention to the instructions in section \ref{others}
regarding figures, tables, acknowledgments, and references.


There will be a strict upper limit of 9 pages for the main text of the initial submission, with unlimited additional pages for citations. 

\section{Headings: first level}
\label{headings}

First level headings are in small caps,
flush left and in point size 12. One line space before the first level
heading and 1/2~line space after the first level heading.

\subsection{Headings: second level}

Second level headings are in small caps,
flush left and in point size 10. One line space before the second level
heading and 1/2~line space after the second level heading.

\subsubsection{Headings: third level}

Third level headings are in small caps,
flush left and in point size 10. One line space before the third level
heading and 1/2~line space after the third level heading.

\section{Citations, figures, tables, references}
\label{others}

These instructions apply to everyone, regardless of the formatter being used.

\subsection{Citations within the text}

Citations within the text should be based on the \texttt{natbib} package
and include the authors' last names and year (with the ``et~al.'' construct
for more than two authors). When the authors or the publication are
included in the sentence, the citation should not be in parenthesis using \verb|\citet{}| (as
in ``See \citet{Hinton06} for more information.''). Otherwise, the citation
should be in parenthesis using \verb|\citep{}| (as in ``Deep learning shows promise to make progress
towards AI~\citep{Bengio+chapter2007}.'').

The corresponding references are to be listed in alphabetical order of
authors, in the \textsc{References} section. As to the format of the
references themselves, any style is acceptable as long as it is used
consistently.

To cite a new paper, first, you need to add that paper's BibTeX information to \verb+APS360_ref.bib+ file and then you can use the \verb|\citep{}| command to cite that in your main document. 

\subsection{Footnotes}

Indicate footnotes with a number\footnote{Sample of the first footnote} in the
text. Place the footnotes at the bottom of the page on which they appear.
Precede the footnote with a horizontal rule of 2~inches
(12~picas).\footnote{Sample of the second footnote}

\subsection{Figures}

All artwork must be neat, clean, and legible. Lines should be dark
enough for purposes of reproduction; art work should not be
hand-drawn. The figure number and caption always appear after the
figure. Place one line space before the figure caption, and one line
space after the figure. The figure caption is lower case (except for
first word and proper nouns); figures are numbered consecutively.

Make sure the figure caption does not get separated from the figure.
Leave sufficient space to avoid splitting the figure and figure caption.

You may use color figures.
However, it is best for the
figure captions and the paper body to make sense if the paper is printed
either in black/white or in color.

\begin{figure}[h]
\begin{center}
\includegraphics[width=0.6\textwidth]{Figs/td-deep-learning.jpg}
\end{center}
\caption{Sample figure caption. Image: ZDNet}
\end{figure}

\subsection{Tables}

All tables must be centered, neat, clean and legible. Do not use hand-drawn
tables. The table number and title always appear before the table. See
Table~\ref{sample-table}.

Place one line space before the table title, one line space after the table
title, and one line space after the table. The table title must be lower case
(except for first word and proper nouns); tables are numbered consecutively.

\begin{table}[t]
\caption{Sample table title}
\label{sample-table}
\begin{center}
\begin{tabular}{ll}
\multicolumn{1}{c}{\bf PART}  &\multicolumn{1}{c}{\bf DESCRIPTION}
\\ \hline \\
Dendrite         &Input terminal \\
Axon             &Output terminal \\
Soma             &Cell body (contains cell nucleus) \\
\end{tabular}
\end{center}
\end{table}


\section{Introduction}


\section{Illustration}


\section{Background \& Related Work}


\section{Data Processing}


\section{Architecture}


\section{Baseline Model}

We will be comparing our model against a Markov chain based text generation algorithm. The specific instance we will derive this from is.
Markov chain based algorithms for text generation were one of the more common methods before neural network based models became a viable option. By starting with a body of text, the algorithm creates a probability map of which words are most likely to follow one another, and through a starting word or phrase, the algorithm recursively finds the most likely word to appear next in a sentence based on the previous word until a desired length is achieved. These algorithms are usually simple and highly performant, however, because they are only able to capture local patterns, Markov based algorithms are limited in scope and have a lack of coherence over longer passages. Comparing our neural network model against a Markov chain algorithm will demonstrate the difference in sophistication we are able to achieve using a more modern approach, and how it changes over various generated passages.


\section{Ethical Considerations}


\section{Project Plan}
\label{project_plan}

\subsection{Communication}

\subsection{Team Norms}


\section{Risk Register}
\label{risk_register}

\subsection{Model does not generate coherent output}

This is possible due to the complexity of generating text, and the potentially large variations in the training data. If our created model is not generating poems that make sense, does not output enough text to be acceptable, or takes too long to generate text, we would first scale back the model to only focus on creating one sentence at a time, and increase the amount of training data it has to work with. Once we have refined our model and see consistent results, we would then increase its size and continuously evaluate its performance as it scales.

\subsection{Team member drops the course}

If a team member decides to drop the course, a meeting will take place where the departing member will go over in detail all of their assigned sections such as: what has been completed, what is in progress, what gave them challenges, etc. Once the team has figured out which parts of the project are affected, the remaining members will meet and see which parts align best with their currently assigned portions and/or expertise. In the event the departing member had a portion that is unrealistic to finish in time, the scope of the project will be re-evaluated.

\subsection{Model is taking too many resources to train}

If the model requires much more time or compute resources to properly train than anticipated, and it is not an issue of improperly constructed code, the team will analyze the individual components of the model to figure out which ones are requiring the largest amount of resources. For example, if the training sets are too large or take too long on our computers, we will first see if we have access to a more powerful computer the training can be performed on. If this is unrealistic or inaccessible, the scope will be re-evaluated in order to have it properly fit within our given resources, which could be reducing complexity of the model or using smaller datasets.

\subsection{Original Scope is too large}

If the realization comes up that we were too ambitious in our original goal, and constraints such as available working time or complexity become large enough issues where in the proposed design is unlikely to be met, each member will share their concerns which could be regarding other commitments such as classes, lack of understanding when it comes to their assigned section, or any other concern that impacts the progress of the project. We will then make any necessary changes to the project design and scope to ensure a working model is delivered.

\subsection{Internal Disagreement about project direction}

If one or more team members brings up any issues they see with the project that could impact progress, a team meeting will take place at which the concerns are communicated to the other members. Each member will have the opportunity to share their opinions on the matter, after which we will deliberate over potential solutions until the entire team feels satisfied with the choices made going forward.


\section{Final instructions}
Do not change any aspects of the formatting parameters in the style files.
In particular, do not modify the width or length of the rectangle the text
should fit into, and do not change font sizes (except perhaps in the
\textsc{References} section; see below). Please note that pages should be
numbered.


\subsubsection*{Author Contributions}
If you'd like to, you may include  a section for author contributions as is done
in many journals. This is optional and at the discretion of the authors.

\subsubsection*{Acknowledgments}
Use unnumbered third level headings for the acknowledgments. All
acknowledgments, including those to funding agencies, go at the end of the paper.

\label{last_page}

\bibliography{APS360_ref}
\bibliographystyle{iclr2022_conference}

\end{document}
